\documentclass{article}
\usepackage[utf8]{inputenc}
\usepackage{amsmath}
\usepackage{amssymb}
\usepackage{enumitem}
\usepackage[margin=1in]{geometry}

\begin{document}

\title{Cognitive Science Practice Test}
\author{}
\date{}
\maketitle

\section*{Instructions}
This is a multiple-choice practice test. For each question, select the best answer from the given options (a), (b), (c), or (d).

\begin{enumerate}[label=\arabic*.]

% Question 1
\item Cognitive science emerged as an interdisciplinary field combining ideas from which areas?
\begin{enumerate}[label=(\alph*)]
    \item Astronomy, sociology, and economics.
    \item Physics, chemistry, and biology.
    \item Psychology, linguistics, mathematical logic, and information-processing models.
    \item Anthropology, geology, and political science.
\end{enumerate}

% Question 2
\item What was a core tenet of Behaviorism in psychology?
\begin{enumerate}[label=(\alph*)]
    \item Speculating about cognition or other unobservable phenomena is scientific.
    \item Psychology should concern itself only with observable behavior.
    \item Focus on internal mental states as primary drivers of action.
    \item All learning is the result of innate knowledge structures.
\end{enumerate}

% Question 3
\item According to cognitive science, what is the single most fundamental idea about organisms?
\begin{enumerate}[label=(\alph*)]
    \item They are primarily driven by unconscious instincts.
    \item They respond mechanically to reinforcers and stimuli.
    \item They are information processors.
    \item Their behavior is solely determined by genetic predispositions.
\end{enumerate}

% Question 4
\item Which type of conditioning involves reflexive, automatic responses?
\begin{enumerate}[label=(\alph*)]
    \item Operant conditioning.
    \item Social learning.
    \item Classical conditioning.
    \item Insight learning.
\end{enumerate}

% Question 5
\item In classical conditioning, what happens to the association between a conditioned stimulus and an unconditioned stimulus?
\begin{enumerate}[label=(\alph*)]
    \item It is weakened.
    \item It is strengthened.
    \item It is eliminated.
    \item It remains neutral.
\end{enumerate}

% Question 6
\item The famous example of conditioning involving dogs salivating to the sound of a bell is associated with which researcher?
\begin{enumerate}[label=(\alph*)]
    \item B.F. Skinner.
    \item Edward Tolman.
    \item Ivan Pavlov.
    \item Karl Lashley.
\end{enumerate}

% Question 7
\item What did the Tolman and Honzik (1930) study on rats navigating mazes seem to demonstrate, contrary to behaviorist expectations?
\begin{enumerate}[label=(\alph*)]
    \item That all learning requires immediate reinforcement.
    \item That rats could only learn through direct punishment.
    \item That reinforcement was not necessary for learning, showing latent learning.
    \item That rats only learned simple stimulus-response connections.
\end{enumerate}

% Question 8
\item What concept was proposed by Tolman, Ritchie, and Kalish (1946) to explain how rats learned to navigate mazes in terms of spatial layout rather than movements?
\begin{enumerate}[label=(\alph*)]
    \item Reflex arcs.
    \item Cognitive maps.
    \item Habituation curves.
    \item Sensory adaptation.
\end{enumerate}

% Question 9
\item What did Tolman's concept of cognitive maps become fundamental for in cognitive science?
\begin{enumerate}[label=(\alph*)]
    \item Explaining behavior solely through conditioning.
    \item Proposing behavior in terms of representations or stored information.
    \item Dismissing internal mental states as unscientific.
    \item Understanding simple stimulus-response connections.
\end{enumerate}

% Question 10
\item Karl Lashley (1951) focused on the problem of explaining complex behavior, particularly challenging the view that movements are linked as a simple chain. What did he suggest instead?
\begin{enumerate}[label=(\alph*)]
    \item That all behavior is random.
    \item That behavior is solely determined by immediate environmental stimuli.
    \item That complex behaviors are a product of prior planning and hierarchical organization.
    \item That learning happens only through classical conditioning.
\end{enumerate}

% Question 11
\item The idea of subconscious information processing, where much behavior is controlled by planning mechanisms below conscious awareness, grew from the work of which researcher?
\begin{enumerate}[label=(\alph*)]
    \item Ivan Pavlov.
    \item Edward Tolman.
    \item Karl Lashley.
    \item B.F. Skinner.
\end{enumerate}

% Question 12
\item Alan Turing formalized the notion of a purely mechanical procedure using what concept?
\begin{enumerate}[label=(\alph*)]
    \item Intuition.
    \item Insight.
    \item Algorithm.
    \item Randomness.
\end{enumerate}

% Question 13
\item What is an idealized computing mechanism devised by Alan Turing, consisting of an infinitely long tape and a machine head?
\begin{enumerate}[label=(\alph*)]
    \item An abacus.
    \item A slide rule.
    \item A Turing Machine.
    \item A quantum computer.
\end{enumerate}

% Question 14
\item What problem did Turing mathematically prove cannot be solved by a purely mechanical procedure, meaning computer programs can get stuck in endless loops?
\begin{enumerate}[label=(\alph*)]
    \item The Traveling Salesperson Problem.
    \item The Decision Problem.
    \item The Halting Problem.
    \item The Encryption Problem.
\end{enumerate}

% Question 15
\item The Church-Turing thesis states that anything that can be done in mathematics by an algorithm can be done by what?
\begin{enumerate}[label=(\alph*)]
    \item A human brain.
    \item A supercomputer.
    \item A Turing machine.
    \item A random number generator.
\end{enumerate}

% Question 16
\item Which linguist proposed transformational grammar in "Syntactic Structures" (1957)?
\begin{enumerate}[label=(\alph*)]
    \item B.F. Skinner.
    \item George Miller.
    \item Noam Chomsky.
    \item Claude Shannon.
\}\end{enumerate}

% Question 17
\item In Chomsky's transformational grammar, what is the distinction between the "deep structure" and the "surface structure" of a sentence?
\begin{enumerate}[label=(\alph*)]
    \item Deep structure is the sounds, surface structure is the meaning.
    \item Deep structure is how it's built from basic constituents, surface structure is the actual word organization.
    \item Deep structure is conscious, surface structure is subconscious.
    \item Deep structure is learned, surface structure is innate.
\end{enumerate}

% Question 18
\item The transformational principles of transformational grammar are examples of what?
\begin{enumerate}[label=(\alph*)]
    \item Random linguistic mutations.
    \item Emotional expressions.
    \item Algorithms.
    \item Purely statistical correlations.
\end{enumerate}

% Question 19
\item Claude E. Shannon's 1948 article "A mathematical theory of communication" marked the emergence of what?
\begin{enumerate}[label=(\alph*)]
    \item Behaviorism.
    \item Information theory.
    \item Psychoanalysis.
    \item Quantum mechanics.
\end{enumerate}

% Question 20
\item George Miller’s (1956) paper "The Magical Number Seven, Plus or Minus Two" proposed what about human sensory systems?
\begin{enumerate}[label=(\alph*)]
    \item They have unlimited channel capacity.
    \item They are all information channels with roughly the same channel capacity.
    \item They are primarily influenced by emotional states.
    \item They operate independently of each other.
\end{enumerate}

% Question 21
\item What strategy did Miller propose to increase the channel capacity of human perceptual systems?
\begin{enumerate}[label=(\alph*)]
    \item Ignoring all incoming information.
    \item Rapid serial processing.
    \item Chunking information.
    \item Increasing the amount of raw data.
\end{enumerate}

% Question 22
\item Donald Broadbent's 1958 book "Perception and Communication" developed one of the first models of how sensory information is processed. What phenomenon did he use to illustrate selective attention?
\begin{enumerate}[label=(\alph*)]
    \item The phantom limb phenomenon.
    \item The cocktail party phenomenon.
    \item The placebo effect.
    \item The Mozart effect.
\end{enumerate}

% Question 23
\item In Broadbent's model, what happens to information after it passes through a short-term store and before semantic interpretation?
\begin{enumerate}[label=(\alph*)]
    \item It is completely discarded.
    \item It passes through a selective filter.
    \item It is immediately sent to long-term memory.
    \item It is amplified for all channels.
\end{enumerate}

% Question 24
\item What was SHRDLU, the program written by Terry Winograd, designed to do?
\begin{enumerate}[label=(\alph*)]
    \item Play chess against human opponents.
    \item Diagnose medical conditions based on symptoms.
    \item Interact with a user in a limited "micro-world" of colored blocks and answer questions about it.
    \item Translate languages in real-time.
\end{enumerate}

% Question 25
\item Unlike earlier "chatterbots" like ELIZA, what did SHRDLU possess?
\begin{enumerate}[label=(\alph*)]
    \item An ability to simulate empathy.
    \item A genuine understanding of the semantics and context of commands.
    \item The capacity to generate original poetry.
    \item A human-like voice interface.
\end{enumerate}

% Question 26
\item SHRDLU exemplified a component-based, top-down approach. What does this mean?
\begin{enumerate}[label=(\alph*)]
    \item It focused only on the output, ignoring internal processes.
    \item It broke down complex tasks into distinct, interacting components.
    \item It relied on a single, undifferentiated processing unit.
    \item It learned purely from random input.
\end{enumerate}

% Question 27
\item Roger Shepard's mental rotation experiments (1971) found a direct, linear relationship between the time taken to solve a problem and what?
\begin{enumerate}[label=(\alph*)]
    \item The color of the figures.
    \item The complexity of the instructions.
    \item The degree of rotation between two figures.
    \item The number of figures presented.
\end{enumerate}

% Question 28
\item According to the sources, how is imagistic representation different from digital representation?
\begin{enumerate}[label=(\alph*)]
    \item Imagistic representation is faster to process.
    \item Imagistic representation has an arbitrary link between symbol and what it represents.
    \item Imagistic representation resembles what it depicts, like a map.
    \item Imagistic representation is always conscious.
\end{enumerate}

% Question 29
\item Stephen Kosslyn's (1973) mental scanning experiments showed that the time it took subjects to confirm a part of a memorized object varied directly with what?
\begin{enumerate}[label=(\alph*)]
    \item The total size of the object.
    \item The distance of the part from their mental "point of focus."
    \item The emotional significance of the object.
    \item The recency of exposure to the object.
\end{enumerate}

% Question 30
\item David Marr's model of the human visual system broke down cognitive systems into three distinct levels of analysis. What are these levels?
\begin{enumerate}[label=(\alph*)]
    \item Input, Output, and Feedback.
    \item Sensory, Perceptual, and Motor.
    \item Computational, Algorithmic, and Implementational.
    \item Conscious, Subconscious, and Unconscious.
\end{enumerate}

% Question 31
\item At Marr's Computational Level, what is the primary focus?
\begin{enumerate}[label=(\alph*)]
    \item How the system is physically built in the brain.
    \item The specific steps (algorithm) used to solve a problem.
    \item The exact information-processing problem the system is designed to solve (what and why).
    \item The historical development of the system.
\end{enumerate}

% Question 32
\item What is the main purpose of human vision, according to Marr, at the computational level?
\begin{enumerate}[label=(\alph*)]
    \item To generate random visual patterns.
    \item To build a description of the shapes and positions of things from images.
    \item To directly control motor movements without conscious awareness.
    \item To process only color information.
\end{enumerate}

% Question 33
\item Marr's Algorithmic Level describes how the visual system achieves 3D understanding through a series of increasingly complex representations. What is the very first step in this series?
\begin{enumerate}[label=(\alph*)]
    \item 3-D Model Representation.
    \item 2.5-D Sketch.
    \item Primal Sketch.
    \item Semantic Interpretation.
\end{enumerate}

% Question 34
\item Research into patients with brain damage by Elizabeth Warrington heavily influenced Marr's thinking at the computational level. What distinct vision problems did she study that led to conclusions about separate processing of shape and object identity?
\begin{enumerate}[label=(\alph*)]
    \item Patients with memory loss only.
    \item Patients with right vs. left parietal damage causing distinct vision problems.
    \item Patients who could not hear but could see perfectly.
    \item Patients with only motor control deficits.
\end{enumerate}

% Question 35
\item The "Two Visual Systems Hypothesis" suggests that information takes different paths based on what information it is. Damage in which brain area is associated with object identification problems?
\begin{enumerate}[label=(\alph*)]
    \item Posterior Parietal Lobe.
    \item Occipital Lobe.
    \item Frontal Lobe.
    \item Temporal Cortex.
\end{enumerate}

% Question 36
\item Which pathway in the "Two Visual Systems Hypothesis" is known as the "what" pathway, responsible for recognizing and identifying objects?
\begin{enumerate}[label=(\alph*)]
    \item The Dorsal Route.
    \item The Ventral Route.
    \item The Auditory Pathway.
    \item The Somatosensory Pathway.
\end{enumerate}

% Question 37
\item Artificial Neural Networks (ANNs) are organized into layers. How are units within these layers typically connected?
\begin{enumerate}[label=(\alph*)]
    \item Each unit connects to other units in the same layer.
    \item Each unit connects to units only in the layer after it, but not before.
    \item Each unit connects to the layer before and after it, but not to other units in the same layer.
    \item Units are randomly connected to any other unit in the network.
\end{enumerate}

% Question 38
\item What property of connections in ANNs enables the network to be trained?
\begin{enumerate}[label=(\alph*)]
    \item Their fixed values.
    \item Their random distribution.
    \item The ability of assigned weights to be modified.
    \item Their purely inhibitory nature.
\end{enumerate}

% Question 39
\item What are "hidden units" in an Artificial Neural Network?
\begin{enumerate}[label=(\alph*)]
    \item Units that are only connected to the output layer.
    \item Units that are only connected to the input layer.
    \item Units that are only connected to units within the same layer.
    \item Units that are only connected to units in other hidden layers.
\end{enumerate}

% Question 40
\item What does fMRI measure to infer brain activity?
\begin{enumerate}[label=(\alph*)]
    \item Direct electrical impulses from neurons.
    \item The temperature changes in brain tissue.
    \item Blood oxygen levels (BOLD signal).
    \item The structural integrity of brain matter.
\end{enumerate}

% Question 41
\item According to Logothetis's experiments, the BOLD signal detected by fMRI is correlated with what aspect of neural activity?
\begin{enumerate}[label=(\alph*)]
    \item Only the firing rate of individual neurons.
    \item The sum of inputs to each neuron (local field potential).
    \item The production of neurotransmitters.
    \item The structural changes in neuron dendrites.
\end{enumerate}

% Question 42
\item What is the brain's most fundamental feature, according to the source?
\begin{enumerate}[label=(\alph*)]
    \item Its large size.
    \item Its weight.
    \item Its chemical composition.
    \item Its connectivity.
\end{enumerate}

% Question 43
\item Which neuroimaging technique involves measuring blood flow using radioactive tracers and has low temporal resolution?
\begin{enumerate}[label=(\alph*)]
    \item fMRI.
    \item EEG.
    \item MEG.
    \item PET.
\end{enumerate}

% Question 44
\item What are the two types of neurites in a neuron?
\begin{enumerate}[label=(\alph*)]
    \item Nucleus and cytoplasm.
    \item Myelin sheath and nodes of Ranvier.
    \item Dendrites and axon.
    \item Cell body and terminal buttons.
\end{enumerate}

% Question 45
\item What happens if the sum of excitatory and inhibitory synapses reaching a neuron's dendrites exceeds the neuron's threshold?
\begin{enumerate}[label=(\alph*)]
    \item The neuron stops functioning.
    \item The neuron fires.
    \item The neuron enters a resting state.
    \item The neuron's weight decreases.
\end{enumerate}

% Question 46
\item Which function cannot be represented by a single layer artificial neural network due to failing to be linearly separable?
\begin{enumerate}[label=(\alph*)]
    \item AND.
    \item OR.
    \item NOT.
    \item XOR.
\end{enumerate}

% Question 47
\item What is the main difference between a single layer network and a multilayer network?
\begin{enumerate}[label=(\alph*)]
    \item Single layer networks process visual information, multilayer networks process auditory.
    \item Multilayer networks contain hidden units that only receive input from other units.
    \item Single layer networks use only excitatory connections, multilayer use inhibitory.
    \item Multilayer networks are always biologically plausible.
\end{enumerate}

% Question 48
\item In multilayer networks, what is the basic idea of the backpropagation algorithm?
\begin{enumerate}[label=(\alph*)]
    \item Error is propagated forward through the network, from input to output.
    \item Error is propagated backward through the network, from the output unit to the hidden units.
    \item Weights are initialized randomly and never updated.
    \item The network learns by receiving external feedback only.
\end{enumerate}

% Question 49
\item What is a key difference between computational neuroscientists and connectionist modelers?
\begin{enumerate}[label=(\alph*)]
    \item Computational neuroscientists only use fMRI data, while connectionists use EEG.
    \item Computational neuroscientists model biological neurons, while connectionists start with generic models to reproduce psychological phenomena.
    \item Connectionists build models from scratch, while computational neuroscientists use existing software.
    \item Computational neuroscientists focus on the macro level, while connectionists focus on the micro level.
\end{enumerate}

% Question 50
\item In a localist network, what does each unit typically code for?
\begin{enumerate}[label=(\alph*)]
    \item A specific brain region.
    \item A general cognitive process.
    \item A specific feature in the input data.
    \item The network's overall output.
\end{enumerate}

% Question 51
\item According to the sources, what is a fundamental feature of neural networks regarding information storage and processing?
\begin{enumerate}[label=(\alph*)]
    \item They maintain a clear distinction between storage and processing.
    \item Information storage and processing are completely separate entities.
    \item There is no clear distinction between information storage and processing.
    \item Information is only stored, never processed.
\end{enumerate}

% Question 52
\item What does the physical symbol system hypothesis propose regarding cognitive processes?
\begin{enumerate}[label=(\alph*)]
    \item They are purely analog.
    \item They work by manipulating symbols using clear, rule-based operations.
    \item They are entirely based on random associations.
    \item They operate independently of any physical substrate.
\end{enumerate}

% Question 53
\item How do neural networks typically differ from symbolic systems in terms of learning?
\begin{enumerate}[label=(\alph*)]
    \item Neural networks are usually fixed and rule-based.
    \item Neural networks cannot adapt through experience.
    \item Neural networks can learn and adapt through experience.
    \item Symbolic systems are better at pattern recognition.
\end{enumerate}

% Question 54
\item What is the definition of neuroscience?
\begin{enumerate}[label=(\alph*)]
    \item The study of observable behavior only.
    \item The study of the nervous system anatomy and physiology.
    \item The study of mathematical logic and algorithms.
    \item The study of human-computer interaction.
\end{enumerate}

% Question 55
\item Which lobe of the brain is primarily responsible for visual processing?
\begin{enumerate}[label=(\alph*)]
    \item Frontal lobe.
    \item Parietal lobe.
    \item Occipital lobe.
    \item Temporal lobe.
\end{enumerate}

% Question 56
\item What is "contralateral organization" in the brain?
\begin{enumerate}[label=(\alph*)]
    \item Information is processed on the same side of the brain it is received from.
    \item The brain processes visual information in the frontal lobe.
    \item Information from one side of the body is mapped onto the opposite side of the brain.
    \item Both hemispheres process information identically.
\end{enumerate}

% Question 57
\item Which brain imaging technique has high temporal resolution (millisecond scale) but poor spatial resolution, helpful in studying perception and attention?
\begin{enumerate}[label=(\alph*)]
    \item PET.
    \item fMRI.
    \item EEG/ERPs.
    \item KESM.
\end{enumerate}

% Question 58
\item What is the main utility of the Knife-Edge Scanning Microscope (KESM)?
\begin{enumerate}[label=(\alph*)]
    \item Measuring real-time brain activity during tasks.
    \item Producing static 3D maps of entire brains for structural and connectivity studies.
    \item Stimulating specific brain areas non-invasively.
    \item Recording the electrical activity of single neurons in live subjects.
\end{enumerate}

% Question 59
\item What are dendrites in a typical neuron primarily responsible for?
\begin{enumerate}[label=(\alph*)]
    \item Carrying the electrical signal down the neuron.
    \item Releasing neurotransmitters into the synapse.
    \item Receiving signals from other neurons.
    \item Containing the nucleus and integrating inputs.
\end{enumerate}

% Question 60
\item Which neurotransmitter is the main inhibitory transmitter in the brain?
\begin{enumerate}[label=(\alph*)]
    \item Acetylcholine.
    \item Dopamine.
    \item GABA.
    \item Glutamate.
\end{enumerate}

% Question 61
\item What is the large bundle of connecting axons that transfers information back and forth between the cerebral hemispheres?
\begin{enumerate}[label=(\alph*)]
    \item The Thalamus.
    \item The Cerebellum.
    \item The Corpus Callosum.
    \item The Brain Stem.
\end{enumerate}

% Question 62
\item What term describes a large cleft or separation between two areas of brain tissue?
\begin{enumerate}[label=(\alph*)]
    \item Gyrus.
    \item Sulcus.
    \item Fissure.
    \item Nucleus.
\end{enumerate}

% Question 63
\item Which lobe is located anteriorly, bounded by the central sulcus and lateral fissure, and is involved in reasoning and planning?
\begin{enumerate}[label=(\alph*)]
    \item Temporal lobe.
    \item Parietal lobe.
    \item Occipital lobe.
    \item Frontal lobe.
\end{enumerate}

% Question 64
\item In the visual system, after initial processing in the primary visual cortex, what are the two main streams that visual input is divided into?
\begin{enumerate}[label=(\alph*)]
    \item Sensory and Motor pathways.
    \item Conscious and Unconscious pathways.
    \item Dorsal and Ventral pathways.
    \item Auditory and Somatosensory pathways.
\end{enumerate}

% Question 65
\item Damage to which visual pathway is typically linked to Visual Agnosias?
\begin{enumerate}[label=(\alph*)]
    \item Dorsal pathway.
    \item Ventral pathway.
    \item Optic nerve.
    \item Reticular activating system.
\end{enumerate}

% Question 66
\item What is Apperceptive Agnosia characterized by?
\begin{enumerate}[label=(\alph*)]
    \item Inability to recognize faces.
    \item Inability to recall names of objects.
    \item Cannot assemble object features into a meaningful whole despite intact basic visual abilities.
    \item Can perceive whole objects but can't name or identify them.
\end{enumerate}

% Question 67
\item What is Prosopagnosia?
\begin{enumerate}[label=(\alph*)]
    \item A type of amnesia.
    \item An inability to recognize sounds.
    \item An inability to recognize faces.
    \item A disorder of motor control.
\end{enumerate}

% Question 68
\item What is the "binding problem" in cognitive science?
\begin{enumerate}[label=(\alph*)]
    \item How different memories are linked together.
    \item How different features of an object processed in separate brain areas are combined into a unified perception.
    \item How language is linked to thought.
    \item How conscious and unconscious processes interact.
\end{enumerate}

% Question 69
\item What hypothesis suggests that an object is represented by the coordinated and phase-locked activity of a constellation of cells?
\begin{enumerate}[label=(\alph*)]
    \item The Grandmother Cell hypothesis.
    \item The Neural Synchrony hypothesis.
    \item The Equipotentiality principle.
    \item The Reflex Arc hypothesis.
\end{enumerate}

% Question 70
\item Which brain structure, located in the hindbrain, controls overall arousal and alertness and is linked to sustaining attention over time?
\begin{enumerate}[label=(\alph*)]
    \item Thalamus.
    \item Superior Colliculus.
    \item Reticular Activating System (RAS).
    \item Cingulate Cortex.
\end{enumerate}

% Question 71
\item What is Hemispatial Neglect?
\begin{enumerate}[label=(\alph*)]
    \item An inability to speak fluently.
    \item Ignoring stimuli or one side of space and body, typically from right hemisphere damage.
    \item A disorder of episodic memory.
    \item Difficulty recognizing familiar faces.
\end{enumerate}

% Question 72
\item Karl Lashley's principle of equipotentiality suggested what about memory?
\begin{enumerate}[label=(\alph*)]
    \item Memory is localized to a single brain area.
    \item All brain areas contribute equally to memory.
    \item Memory is primarily stored in the spinal cord.
    \item Memory is formed only through classical conditioning.
\end{enumerate}

% Question 73
\item Which brain area is key to memory consolidation (transferring to long-term memory), with damage causing anterograde amnesia?
\begin{enumerate}[label=(\alph*)]
    \item Amygdala.
    \item Basal ganglia.
    \item Hippocampus.
    \item Cerebellum.
\end{enumerate}

% Question 74
\item What type of memory is supported by the basal ganglia?
\begin{enumerate}[label=(\alph*)]
    \item Declarative memories (facts, events).
    \item Episodic memories.
    \item Procedural knowledge (skills, habits).
    \item Semantic memories.
\end{enumerate}

% Question 75
\item According to Herbert Simon's computational theory of emotions, what do emotions primarily function as within the CNS?
\begin{enumerate}[label=(\alph*)]
    \item Long-term storage units for past experiences.
    \item Mechanisms for generating random behaviors.
    \item Interrupt mechanisms that substitute new goals and behaviors.
    \item Permanent states of physiological arousal.
\end{enumerate}

% Question 76
\item Paul Ekman's basic emotion theory makes two main claims, including that particular facial expressions are universally associated with particular emotions. Which of the following is NOT on his list of basic emotions?
\begin{enumerate}[label=(\alph*)]
    \item Happiness.
    \item Surprise.
    \item Envy.
    \item Fear.
\end{enumerate}

% Question 77
\item What are the two most popular dimensions used in the "affective space" to distinguish different affective states?
\begin{enumerate}[label=(\alph*)]
    \item Memory and attention.
    \item Learning and perception.
    \item Valence and Arousal.
    \item Cognition and behavior.
\end{enumerate}

% Question 78
\item What subcortical area forms part of the limbic system and has been shown to be especially important for fear experience and recognition, as seen in the case of S.M.?
\begin{enumerate}[label=(\alph*)]
    \item Hippocampus.
    \item Thalamus.
    \item Amygdala.
    \item Hypothalamus.
\end{enumerate}

% Question 79
\item What are the three core ideas of Bayesianism as a framework in cognitive science?
\begin{enumerate}[label=(\alph*)]
    \item Belief is fixed, probabilities are irrelevant, learning is unconscious.
    \item Belief comes in degrees, degrees conform to probability rules, learning occurs via Bayesian updating.
    \item Belief is absolute, learning is only through reinforcement, perception is direct.
    \item Belief is always conscious, learning is always supervised, evidence is always perfect.
\end{enumerate}

% Question 80
\item In Bayesian terms, how is "Posterior Probability" calculated?
\begin{enumerate}[label=(\alph*)]
    \item Posterior = Likelihood + Prior.
    \item Posterior = Evidence / Likelihood.
    \item Posterior = Likelihood $\times$ Prior / Evidence Probability.
    \item Posterior = Prior / Likelihood.
\end{enumerate}

% Question 81
\item What does the "Expected Utility Theory" help to do?
\begin{enumerate}[label=(\alph*)]
    \item Predict emotional responses.
    \item Explain the origin of consciousness.
    \item Make optimal decisions by weighing probabilities and preferences (utilities).
    \item Measure brain activity during perception.
\end{enumerate}

% Question 82
\item What is the "Modularity of Mind" hypothesis proposed by Jerry Fodor?
\begin{enumerate}[label=(\alph*)]
    \item The mind is a single, undifferentiated processing unit.
    \item The mind has special modular subsystems to handle distinct tasks like perception or language.
    \item All cognitive processes are non-modular and distributed.
    \item The mind learns only through direct sensory experience.
\end{enumerate}

% Question 83
\item According to Fodor, what are key properties of central processing, as opposed to modular subsystems?
\begin{enumerate}[label=(\alph*)]
    \item Domain-specific and encapsulated.
    \item Mandatory and fast.
    \item Quinean (beliefs interconnected) and Isotropic (any info affects evaluation).
    \item Fixed neural architectures.
\end{enumerate}

% Question 84
\item What is the "Massive Modularity Hypothesis"?
\begin{enumerate}[label=(\alph*)]
    \item The human mind consists of a single, highly flexible module.
    \item The human mind consists of a large number of domain-specific modules, evolved to solve ancestral problems.
    \item Modules are only found in non-human animals.
    \item All learning happens through classical conditioning.
\end{enumerate}

% Question 85
\item The "Cheater Detection Module" is based on findings related to reasoning with conditional rules, particularly which task?
\begin{enumerate}[label=(\alph*)]
    \item The Prisoner's Dilemma.
    \item The Stroop Effect.
    \item The Wason Selection Task.
    \item The Marshmallow Test.
\end{enumerate}

% Question 86
\item What kind of cognitive architecture is ACT-R (Adaptive Control of Thought)?
\begin{enumerate}[label=(\alph*)]
    \item Purely symbolic.
    \item Purely subsymbolic.
    \item A hybrid model that combines symbolic and subsymbolic processes.
    \item A model based solely on brain anatomy.
\end{enumerate}

% Question 87
\item In ACT-R, what form does "Declarative memory" take?
\begin{enumerate}[label=(\alph*)]
    \item Production rules.
    \item Action schemas.
    \item Symbolic chunks.
    \item Neural network weights.
\end{enumerate}

% Question 88
\item What is the "principle of segregation" in brain mapping?
\begin{enumerate}[label=(\alph*)]
    \item The idea that all brain regions function as one integrated unit.
    \item The idea that the cerebral cortex is divided into segregated areas with distinct neuronal populations.
    \item The method of connecting different brain regions through surgery.
    \item The study of brain damage effects on behavior.
\end{enumerate}

% Question 89
\item What is anatomical connectivity in brain mapping?
\begin{enumerate}[label=(\alph*)]
    \item How brain regions function during specific cognitive tasks.
    \item How brain regions relate to each other physically.
    \item The statistical correlation of activity levels between brain regions.
    \item The conscious awareness of brain activity.
\end{enumerate}

% Question 90
\item Which type of resolution in brain mapping refers to the ability to distinguish or capture changes or events that occur over time?
\begin{enumerate}[label=(\alph*)]
    \item Spatial resolution.
    \item Temporal resolution.
    \item Functional resolution.
    \item Structural resolution.
\end{enumerate}

% Question 91
\item What is the "locus-of-selection problem" in attention research?
\begin{enumerate}[label=(\alph*)]
    \item Determining the optimal number of attention tasks.
    \item Determining whether attention is an early or late selection phenomenon.
    \item Identifying the brain region responsible for attention.
    \item Measuring the total capacity of attention.
\end{enumerate}

% Question 92
\item What does "functional connectivity" in neuroimaging refer to?
\begin{enumerate}[label=(\alph*)]
    \item The causal flow of information between brain areas.
    \item The physical wiring diagram of the brain.
    \item Statistical correlations between levels of activity in physically separate parts of the brain.
    \item The ability of a brain region to perform a specific function.
\end{enumerate}

% Question 93
\item How does "effective connectivity" differ from "functional connectivity" in neuroimaging?
\begin{enumerate}[label=(\alph*)]
    \item Effective connectivity measures physical distance, while functional measures activation.
    \item Effective connectivity measures statistical correlations, while functional measures causal influence.
    \item Effective connectivity measures how neural systems interact and information flows causally, while functional is statistical.
    \item Effective connectivity is a hypothetical concept, while functional connectivity is directly observable.
\end{enumerate}

% Question 94
\item According to Fodor, what does learning a language involve besides mastering rules?
\begin{enumerate}[label=(\alph*)]
    \item Only memorizing words without understanding their meaning.
    \item Forming hypotheses about word and sentence meanings, testing them, and revising as needed.
    \item Relying solely on unconscious imitation.
    \item Being born with complete linguistic knowledge.
\end{enumerate}

% Question 95
\item What is the "Poverty of the Stimulus" argument used by Fodor and Chomsky to support the idea of innate linguistic knowledge?
\begin{enumerate}[label=(\alph*)]
    \item Children are exposed to too much linguistic information.
    \item Children are not exposed to enough information to learn a language from scratch.
    \item Language learning is a purely cultural phenomenon.
    \item Animals can learn human language with sufficient training.
\end{enumerate}

% Question 96
\item In the context of child language learning of past tense, what is "overregularization"?
\begin{enumerate}[label=(\alph*)]
    \item Consistently using only irregular verbs correctly.
    \item Applying the "-ed" ending to irregular verbs (e.g., "gived" instead of "gave").
    \item Learning new verbs too slowly.
    \item Speaking in full sentences from birth.
\end{enumerate}

% Question 97
\item The Rumelhart & McClelland (1986) connectionist model of tense learning used which learning algorithm?
\begin{enumerate}[label=(\alph*)]
    \item Backpropagation.
    \item Gradient Descent.
    \item Perceptron Convergence Rule.
    \item Hebbian Learning (unsupervised).
\end{enumerate}

% Question 98
\item What is the first step in understanding a language, according to the transitional probabilities model in developmental linguistics?
\begin{enumerate}[label=(\alph*)]
    \item Learning complex grammar rules.
    \item Dividing speech into individual words (word segmentation).
    \item Identifying abstract concepts.
    \item Developing a complete vocabulary.
\end{enumerate}

% Question 99
\item What is "phenomenal consciousness" (P-consciousness) according to Ned Block?
\begin{enumerate}[label=(\alph*)]
    \item A state that is poised for direct control of thought and action.
    \item The ability to report on one's mental states.
    \item Experience, such as hearing, smelling, or feeling emotions.
    \item The logical processing of information.
\end{enumerate}

% Question 100
\item According to David Chalmers, what is the "hard problem" of consciousness?
\begin{enumerate}[label=(\alph*)]
    \item Explaining how cognitive systems integrate information.
    \item Explaining how attention gets focused.
    \item Explaining experience, the subjective aspect ("what it is like").
    \item Explaining the deliberate control of behavior.
\end{enumerate}

\end{enumerate}

\clearpage

\section*{Answer Sheet}

\begin{enumerate}[label=\arabic*.]
    \item c
    \item b
    \item c
    \item c
    \item b
    \item c
    \item c
    \item b
    \item b
    \item c
    \item c
    \item c
    \item c
    \item c
    \item c
    \item c
    \item b
    \item c
    \item b
    \item b
    \item c
    \item b
    \item b
    \item c
    \item b
    \item b
    \item c
    \item c
    \item b
    \item c
    \item c
    \item b
    \item c
    \item b
    \item d
    \item b
    \item c
    \item c
    \item c
    \item c
    \item b
    \item d
    \item d
    \item c
    \item b
    \item d
    \item b
    \item b
    \item b
    \item c
    \item c
    \item b
    \item c
    \item b
    \item c
    \item c
    \item c
    \item b
    \item c
    \item c
    \item c
    \item c
    \item d
    \item c
    \item b
    \item c
    \item c
    \item b
    \item b
    \item c
    \item b
    \item b
    \item c
    \item c
    \item c
    \item c
    \item c
    \item c
    \item b
    \item c
    \item c
    \item b
    \item c
    \item c
    \item c
    \item d
    \item b
    \item b
    \item b
    \item b
    \item c
    \item c
    \item b
    \item c
    \item c
    \item b
    \item b
    \item b
    \item c
    \item c
\end{enumerate}

\end{document}